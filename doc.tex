\documentclass[a4paper,10pt]{book}
\usepackage[utf8]{inputenc}
\input{encabezado.tex}
\input{tikzlibrarybayesnet.code.tex}
\usepackage{physics}
\newcommand*\diff{\mathop{}\!\mathrm{d}}

\newif\ifen
\newif\ifes
\newcommand{\en}[1]{\ifen#1\fi}
\newcommand{\es}[1]{\ifes#1\fi}
\estrue


%opening
\title{\huge Habilidad y evolución cultural}
\author{Gustavo Landfried}

\begin{document}

\maketitle

\tableofcontents

\chapter*{Abstract}


\chapter{Conocimiento empírico}

\epigraph{El conocimiento empírico no es más que principio de incertidumbre compatible con la evidencia formal y empíricia}{}

La ciencia es una institución humana que tiene pretención de verdad, esto es de formular proposiciones que valgan para todas las personas, tanto intercultural como intersubjetivamente.
Las ciencias formales validan sus proposiciones mediante teoremas, resultados derivados de aplicar las reglas internas a un sistema axiomático cerrado.
% Las ciencias de la computación nacen como una rama de las matemáticas aplicadas, y desarrolla en el transcurso de su historia orientaciones tanto de ciencias formales (algoritmos, lógica y computabilidad) y como de ciencias empíricas (simulación, inteligencia artificial).
% A diferencia de las ciencia formales, 
Las ciencias empíricas deben validar sus proposiciones dentro de sistemas abiertos, lo que impone siempre un grado de incertidumbre asociada.
¿Cuál es entonces la fuente de validez del conocimiento empírico?

Supongamos que tenemos 3 cajas y sabemos que detrás de una hay un regalo.
Una posible distribuión de creencias es:

\begin{figure}[H]
\centering
\tikz{ %
        
         \node[factor, minimum size=1cm] (p1) {} ;
         \node[factor, minimum size=1cm, xshift=1.5cm] (p2) {} ;
         \node[factor, minimum size=1cm, xshift=3cm] (p3) {} ;
         \node[const, above=of p1, yshift=.15cm] (fp1) {$1/10$};
         \node[const, above=of p2, yshift=.15cm] (fp2) {$8/10$};
         \node[const, above=of p3, yshift=.15cm] (fp3) {$1/10$};
        } 
\end{figure}

lo que representa una preferencia parcial por la caja del medio.
Pero si de verdad no tenemos ninguna información respecto de dónde está el regalo, no hay motivos para tener preferencia por ninguna de las opciones, lo que sin lugar a dudas nos hará estar de acuerdo en la siguiente distribución de creencias.

\begin{figure}[H]
\centering
\tikz{ %
        
         \node[factor, minimum size=1cm] (p1) {} ;
         \node[factor, minimum size=1cm, xshift=1.5cm] (p2) {} ;
         \node[factor, minimum size=1cm, xshift=3cm] (p3) {} ;
         \node[const, above=of p1, yshift=.15cm] (fp1) {$1/3$};
         \node[const, above=of p2, yshift=.15cm] (fp2) {$1/3$};
         \node[const, above=of p3, yshift=.15cm] (fp3) {$1/3$};
        } 
\end{figure}

Este tipo de distribuciones de creencias, que permiten el acuerdo intersubjetivo, la vamos a llamar \textbf{creencia honesta}.
Las creencias honestas son las que maximizan incertidumbre, siendo coherente con la información disponible.
En este caso, en el que no tenemos información previa, la obtuvimos dividiendo la creencia en partes iguales.
Éste es un viejo principio conocido como ``de indiferencia''.
¿Pero cómo hacemos para actualizar las creencias de forma honesta cuando recibimos nueva información?

\section{Selección de modelos}



\chapter{TrueSkill Through Time}

\chapter{Las implementaciones en Julia, Python y R}

\chapter{Efecto de los equipos sobre el aprendizaje (faithfull-sinergia)}

\chapter{Efecto de la topología sobre el aprendizaje.}

\chapter{Sistema de estimación para el juego Go (AAGo).}


\end{document}
